\documentclass{amsart}

\usepackage{hyperref}
\usepackage{xcolor}
\usepackage{verbatim}
\usepackage{tikz}
\usetikzlibrary{shapes,arrows,calc,positioning}

\newcommand{\type}[1]{{\tt #1}}
\newcommand{\code}[1]{{\tt #1}}
\newcommand{\metaphor}[1]{\href{http://categoricaldata.net/metaphor/#1}{\tt #1}}

\title{User Guide}
\author{Scott Morrison and David Spivak}

\begin{document}

\maketitle

\section{TODO}
\begin{itemize}
\item Write this TODO list.
\item Write some 'recipes'; essentially whatever is currently the most impressive demo.
\end{itemize}

\subsection{Metaphor}
\begin{itemize}
\item The Metaphor section is getting ahead of reality; add trello TODO items.
\end{itemize}
\subsection{Scala}
\begin{itemize}
\item Have the \verb+\type{...}+ command actually link directly to the Scaladoc.
\item Explain about \code{objectsAtLevel}.
\item Draw more of the \type{Functor} type hierarchy.
\end{itemize}

\section{Recipes}

\section{Metaphor}
The Metaphor server (running at \url{http://categoricaldata.net/metaphor}) provides an interface to the underlying Scala library. It works both as a user interface accessible through a web browser and as a RESTful web service, transmitting objects as JSON data. 

The primary objects in Metaphor are \emph{ontologies}, \emph{datasets} and \emph{translations}. Nearly all of the URLs provided by the Metaphor server correspond to one of these three types of objects.  Every ontology, dataset or ontology has a JSON representation (a quasi-human readable, machine parseable format, c.f. \S \ref{sec:JSON}).  If a client requests such a URL, with the header \code{Accept: application/json}, the server will respond with the JSON representation of the corresponding object. If you load such a URL in a web browser (which will typically provide a header like \code{Accept: text/html}) then a human readable web page displaying the object will be returned instead. (Note that this means you can always copy and paste a Metaphor URL from your browser into any input field, or pass it as an argument just as if it were pure JSON.)

\subsection{Examples}
Currently the Metaphor server provides a small supply of example data. See
\begin{itemize}
\item \metaphor{examples/ontologies}
\item \metaphor{examples/datasets}
\item \metaphor{examples/translations}
\end{itemize}
for lists of examples, or go directly to \metaphor{examples/translations/Skip/5/3}.

\subsection{Computations}
The Metaphor server currently supports computation of the three basic data functors (pullback, left and right pushforward), via requests to
\begin{itemize}
\item \metaphor{compute/pullback}
\item \metaphor{compute/leftPushforward}
\item \metaphor{compute/rightPushforward}
\end{itemize}
Each of these requests requires two parameters, \code{translation} and \code{dataset}. The value of these parameters may either be a JSON encoded object or an encoded URL. If it is a URL, the Metaphor server will make the necessary requests to obtain the desired object. (In fact, it might take some shortcuts, if it's a URL pointing to the same Metaphor server.)

Requesting these URLs with missing or invalid parameters from a browser will produce a form asking for further input.

\subsection{External URLs}
It's perfectly possible to store ontologies, translations and datasets outside of the Metaphor server. Any URL that resolves to a JSON encoded object will do.

\subsection{JSON formats}
\label{sec:JSON}

\section{Scala}

\subsection{Class hierarchies}
This section contains an overview of the important types in the Scala library. You should read this section in conjunction with the \href{}{Scaladocs}.

The three most important types are
\begin{itemize}
\item {\color{gray}\type{net.metaphor.api.}}\type{Ontology}

All database schemas have type \type{Ontology}. 

\item {\color{gray}\type{net.metaphor.api.}}\type{Translation}

A \type{Translation} is a functor between two \type{Ontology}s

\item {\color{gray}\type{net.metaphor.api.}}\type{Ontology\#Dataset}

A \type{Dataset} is a functor from an \type{Ontology} to \type{Set}. 
(Recall in Scala the \# denotes an inner class --- thus every \type{Dataset} is attached to a particular \type{Ontology} instance.) 
\end{itemize}

\type{Ontology} is a subtype of a long sequence of more general classes of categories, illustrated in Figure \ref{fig:ontology-hierarchy}.

\tikzset{type/.style={rectangle, rounded corners, draw, fill=red!20, node distance=3cm, inner sep=5pt, align=left}}
\tikzset{companion/.style={rectangle, rounded corners, draw, fill=green!20}}

\begin{figure}[ht]    
\begin{tikzpicture}
\tikzset{type/.append style={text width=8.5cm}, anchor=north}

% the Category hierarchy
\node[type] (Category) {
\vspace{-0.3cm}
\begin{verbatim}
trait Category {
    type O
    type M
    
    def identity(o: O): M
    def source(m: M): O
    def target(m: M): O
    def compose(m1: M, m2: M): M
    ...
}
\end{verbatim}
};
\node[type] at ($(Category.south)+(0,-1)$) (SmallCategory) {
\vspace{-0.3cm}
\begin{verbatim}
trait SmallCategory {
    trait FunctorToSet { ... }
    ...
}
\end{verbatim}
};
\node[type] at ($(SmallCategory.south)+(0,-1)$) (LocallyFinitelyGeneratedCategory) {
\vspace{-0.3cm}
\begin{verbatim}
trait LocallyFinitelyGeneratedCategory { 
    type G
    override type M = PathEquivalenceClass
    def pathEquality(p1: Path, p2: Path): Boolean
    
    override def compose(m1: M, m2: M) = ...
    
    def objectsAtLevel(k: Int): List[O]
    val minimumLevel: Int
    def generators(s: O, t: O): List[G]
    ...
}
\end{verbatim}
};
\node[type] at ($(LocallyFinitelyGeneratedCategory.south)+(0,-1)$) (FinitelyGeneratedCategory) {
\vspace{-0.3cm}
\begin{verbatim}
trait FinitelyGeneratedCategory {
    val maximumLevel: Int
    ...   
}
\end{verbatim}
};
\node[type] at ($(FinitelyGeneratedCategory.south)+(0,-1)$) (FinitelyPresentedCategory) {
\vspace{-0.3cm}
\begin{verbatim}
trait FinitelyPresentedCategory {
    def relations(s: O, t: O): List[(Path, Path)]
    ...
}
\end{verbatim}
};
\node[type] (Ontology) at ($(FinitelyPresentedCategory.south)+(0,-1)$)  {
\vspace{-0.3cm}
\begin{verbatim}
trait Ontology { ... }
\end{verbatim}
};
\node[companion,text width=4cm,anchor=west] (Ontology-companion) at ($(Ontology.east)+(1,0)$) {
\vspace{-0.3cm}
\begin{verbatim}
object Ontology { ... }
\end{verbatim}
};

% inheritances
\draw[->] (Ontology) -- (FinitelyPresentedCategory);
\draw[->] (FinitelyPresentedCategory) -- (FinitelyGeneratedCategory);
\draw[->] (FinitelyGeneratedCategory) -- (LocallyFinitelyGeneratedCategory);
\draw[->] (LocallyFinitelyGeneratedCategory) -- (SmallCategory);
\draw[->] (SmallCategory) -- (Category);
\draw[dashed] (Ontology) -- (Ontology-companion);
\end{tikzpicture}
\caption{The \type{Ontology} type hierarchy.}
\label{fig:ontology-hierarchy}
\end{figure}

\subsection{Categories}
At the top of the hierarchy of categories is \type{Category}. It contains two abstract type members, \code{O} and \code{M}, which represent the types of objects and morphisms for the category. (If a type has an abstract type member, it cannot be instantiated --- some subtype will override these type members, specifying concrete types.) Essentially the only other functionality in \type{Category} are the methods \code{identity}, \code{source}, \code{target} and \code{compose}, which provide the basic operations on objects and morphisms.

Below \type{Category} we have \type{SmallCategory}. For our purposes, a \type{SmallCategory} is a category for which we can talk about the category of functors to \type{Set}. In particular, \type{SmallCategory} contains an inner type \type{FunctorToSet}. (This will eventually be specialized to the inner type \type{Dataset} in \type{Ontology}.)

Below \type{SmallCategory} we have \type{LocallyFinitelyGeneratedCategory}. Mathematically, a locally finitely generated category is a category with a finite set of `generators' between each pair of objects, such that every morphism can be obtained by composing some sequence of generators. Moreover, we insist that the set of generators with a fixed source but arbitrary target is finite (even when there are infinitely many objects), and similarly for a fixed target.

In Scala, we implement this by introducing a new abstract type member \code{G} to represent generators, and override the type \code{M}, defining it once and for all to be \code{PathEquivalenceClass}. Further, \type{LocallyFinitelyGeneratedCategory} provides definitions of \code{identity}, \code{source}, \code{target} and \code{compose}, the basic operations from \type{Category}. An implementation of a \type{LocallyFinitelyGeneratedCategory} must provide a number of new methods, instead. The most important of these is \code{def generators(s: O, t: O): List[G]}, specifying the generators between two objects. Further, the category must provide a method \code{def pathEquality(p1: Path, p2: Path): Boolean}, which determines whether two compositions of generators are equal. (Override the method \code{def pathHashCode(p: Path): Int} is highly recommended as well.)

Below \type{LocallyFinitelyGeneratedCategory} we have \type{FinitelyGeneratedCategory}. Now we insist that there are finitely many objects.

Below \type{FinitelyGeneratedCategory} we have \type{FinitelyPresentedCategory}, which provides a method \code{def relations(s: O, t: O): List[(Path, Path)]}. In principle at least, \type{FinitelyPresentedCategory} could provide an implementation of the method \code{pathEquality}, but since this is potentially a hard (!) problem we defer this to traits which use particular strategies to decide path equality.

Finally, an \type{Ontology} inherits from \type{FinitelyPresentedCategory}.

\subsection{Functors}
The basic \type{Functor} type is
\begin{verbatim}
trait Functor {
    val source: Category
    val target: Category
    
    def onObjects(source.O): target.O
    def onMorphisms(source.M): target.M
}
\end{verbatim}

It also comes with some convenience \code{apply} methods, so if \code{F} is a functor we can simply write \code{F(o)} or \code{F(m)} to apply it to an object or morphism. Implementations, however, should override \code{onObjects} and \code{onMorphisms}.

The type hierarchy is quite complicated, and two-dimensional rather than linear! Most of the relevant types are traits contained in the object \type{Functor}. In particular, we have
\begin{itemize}
\item \type{Functor.withSmallSource}
\item \type{Functor.withLocallyFinitelyGeneratedSource}
\item \type{Functor.withFinitelyGeneratedSource}
\item \type{Functor.withFinitelyPresentedSource}
\end{itemize}
(each inheriting from the last) and 
\begin{itemize}
\item \type{Functor.withSmallTarget}
\item \type{Functor.withLocallyFinitelyGeneratedTarget}
\item \type{Functor.withFinitelyGeneratedTarget}
\item \type{Functor.withFinitelyPresentedTarget}
\end{itemize}
along with the combined types \code{Functor.withXXXSource.withYYYTarget}, for each of the possible values of \code{XXX} and \code{YYY}. This two-dimensional hierarchy, along with the most important features of some types, is shown in Figure \ref{fig:functor-hierarchy}.

\type{Functor.withLocallyFinitelyGeneratedSource} (and hence all its descendants) provides an implementation of \code{onMorphisms}, but has a new abstract method \code{def onGenerators(source.G): target.M} which must be implemented by instances.

See below in \S \ref{sec:datafunctors} for a description of where in this type hierarchy the various data functors (pullback, left and right pushforwards) are defined.


\begin{figure}[ht]
\begin{tikzpicture}[rotate=90,scale=0.85]
\tikzset{type/.append style={text width=4cm}, anchor=north east, transform shape}
\node[type] (Functor) at (0,0) {
\vspace{-0.3cm}
\begin{verbatim}
Functor
\end{verbatim}
};
\node[type] (Functor.ss) at (5,0) {
\vspace{-0.3cm}
\begin{verbatim}
Functor
  .withSmallSource
\end{verbatim}
};
\node[type] (Functor.lfgs) at (10,0) {
\vspace{-0.3cm}
\begin{verbatim}
Functor
  .withLocallyFinitely
     GeneratedSource
\end{verbatim}
};
\node[type] (Functor.fgs) at (15,0) {
\vspace{-0.3cm}
\begin{verbatim}
Functor
  .withFinitely
     GeneratedSource
\end{verbatim}
};
\node[type] (Functor.fps) at (20,0) {
\vspace{-0.3cm}
\begin{verbatim}
Functor
  .withFinitely
     PresentedSource
\end{verbatim}
};
\node[type] (Functor.st) at (0,-3) {
\vspace{-0.3cm}
\begin{verbatim}
Functor
  .withSmallTarget
\end{verbatim}
};
\node[type] (Functor.lfgt) at (0,-6) {
\vspace{-0.3cm}
\begin{verbatim}
Functor
  .withLocallyFinitely
    GeneratedTarget
\end{verbatim}
};
\node[type] (Functor.fgt) at (0,-9) {
\vspace{-0.3cm}
\begin{verbatim}
Functor
  .withFinitely
    GeneratedTarget
\end{verbatim}
};
\node[type] (Functor.fpt) at (0,-12) {
\vspace{-0.3cm}
\begin{verbatim}
Functor
  .withFinitely
    PresentedTarget
\end{verbatim}
};
\end{tikzpicture}
\caption{The \type{Functor} type hierarchy.}
\label{fig:functor-hierarchy}
\end{figure}

\subsection{The category of \type{Set}s}

\subsection{Functors to \type{Set}}
\label{sec:functorstoset}

\subsection{Data functors}
\label{sec:datafunctors}


\end{document}