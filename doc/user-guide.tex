\documentclass{amsart}

\usepackage{hyperref}
\usepackage{xcolor}
\usepackage{verbatim}
\usepackage{tikz}
\usetikzlibrary{shapes,arrows,calc,positioning}

\newcommand{\type}[1]{{\tt #1}}
\newcommand{\code}[1]{{\tt #1}}


\title{User Guide}
\author{Scott Morrison and David Spivak}

\begin{document}

\maketitle

\section{Scala}

\subsection{Class hierarchies}
This section contains an overview of the important types in the Scala library. You should read this section in conjunction with the \href{}{Scaladocs}.

The three most important types are
\begin{itemize}
\item {\color{gray}\type{net.metaphor.api.}}\type{Ontology}

All database schemas have type \type{Ontology}. 

\item {\color{gray}\type{net.metaphor.api.}}\type{Translation}

A \type{Translation} is a functor between two \type{Ontology}s

\item {\color{gray}\type{net.metaphor.api.}}\type{Ontology\#Dataset}

A \type{Dataset} is a functor from an \type{Ontology} to \type{Set}. 
(Recall in Scala the \# denotes an inner class --- thus every \type{Dataset} is attached to a particular \type{Ontology} instance.) 
\end{itemize}

\type{Ontology} is a subtype of a long sequence of more general classes of categories, illustrated in Figure \ref{fig:ontology-hierarchy}.

\tikzset{type/.style={rectangle, rounded corners, draw, fill=red!20, node distance=3cm, inner sep=5pt, align=left}}
\tikzset{companion/.style={rectangle, rounded corners, draw, fill=green!20}}

\begin{figure}[ht]    
\begin{tikzpicture}
\tikzset{type/.append style={text width=8.5cm}, anchor=north}

% the Category hierarchy
\node[type] (Category) {
\vspace{-0.3cm}
\begin{verbatim}
trait Category {
    type O
    type M
    
    def identity(o: O): M
    def source(m: M): O
    def target(m: M): O
    def compose(m1: M, m2: M): M
    ...
}
\end{verbatim}
};
\node[type] at ($(Category.south)+(0,-1)$) (SmallCategory) {
\vspace{-0.3cm}
\begin{verbatim}
trait SmallCategory {
    trait FunctorToSet { ... }
    ...
}
\end{verbatim}
};
\node[type] at ($(SmallCategory.south)+(0,-1)$) (LocallyFinitelyGeneratedCategory) {
\vspace{-0.3cm}
\begin{verbatim}
trait LocallyFinitelyGeneratedCategory { 
    type G
    override type M = PathEquivalenceClass
    def pathEquality(p1: Path, p2: Path): Boolean
    
    override def compose(m1: M, m2: M) = ...
    
    def objectsAtLevel(k: Int): List[O]
    val minimumLevel: Int
    def generators(s: O, t: O): List[G]
    ...
}
\end{verbatim}
};
\node[type] at ($(LocallyFinitelyGeneratedCategory.south)+(0,-1)$) (FinitelyGeneratedCategory) {
\vspace{-0.3cm}
\begin{verbatim}
trait FinitelyGeneratedCategory {
    val maximumLevel: Int
    ...   
}
\end{verbatim}
};
\node[type] at ($(FinitelyGeneratedCategory.south)+(0,-1)$) (FinitelyPresentedCategory) {
\vspace{-0.3cm}
\begin{verbatim}
trait FinitelyPresentedCategory {
    def relations(s: O, t: O): List[(Path, Path)]
    ...
}
\end{verbatim}
};
\node[type] (Ontology) at ($(FinitelyPresentedCategory.south)+(0,-1)$)  {
\vspace{-0.3cm}
\begin{verbatim}
trait Ontology { ... }
\end{verbatim}
};
\node[companion,text width=4cm,anchor=west] (Ontology-companion) at ($(Ontology.east)+(1,0)$) {
\vspace{-0.3cm}
\begin{verbatim}
object Ontology { ... }
\end{verbatim}
};

% inheritances
\draw[->] (Ontology) -- (FinitelyPresentedCategory);
\draw[->] (FinitelyPresentedCategory) -- (FinitelyGeneratedCategory);
\draw[->] (FinitelyGeneratedCategory) -- (LocallyFinitelyGeneratedCategory);
\draw[->] (LocallyFinitelyGeneratedCategory) -- (SmallCategory);
\draw[->] (SmallCategory) -- (Category);
\draw[dashed] (Ontology) -- (Ontology-companion);
\end{tikzpicture}
\label{fig:ontology-hierarchy}
\caption{The \type{Ontology} type hierarchy.}
\end{figure}

\end{document}